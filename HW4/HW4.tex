\documentclass[letterpaper,10pt,titlepage]{article}

\usepackage{graphicx}                                        
\usepackage{amssymb}                                         
\usepackage{amsmath}                                         
\usepackage{amsthm}                                          

\usepackage{alltt}                                           
\usepackage{float}
\usepackage{color}
\usepackage{url}

\usepackage{balance}
%\usepackage[TABBOTCAP, tight]{subfigure}
\usepackage{enumitem}
%\usepackage{pstricks, pst-node}

\usepackage{geometry}
\geometry{textheight=8.5in, textwidth=6in}

\newcommand{\cred}[1]{{\color{red}#1}}
\newcommand{\cblue}[1]{{\color{blue}#1}}

\usepackage{hyperref}
\usepackage{geometry}

\def\name{Ryan Crane and Shawn Cross}

%pull in the necessary preamble matter for pygments output
%\input{pygments.tex}

%% The following metadata will show up in the PDF properties
\hypersetup{
  colorlinks = true,
  urlcolor = black,
  pdfauthor = {\name},
  pdfkeywords = {cs444 ``operating systems'' files filesystem I/O},
  pdftitle = {CS 444 Project 1: UNIX File I/O},
  pdfsubject = {CS 444 Project 1},
  pdfpagemode = UseNone
}

\begin{document}

\title{Assignment 4}
\date{December 1, 2017}
\author{Shawn Cross, Ryan Crane}
\maketitle

\section{Plan for Implementation}

For this project our plan is to try and first find a way to apply the 
best fit solution and then we will use the system calls to determine the 
fragmentation that we are getting with this solution. We will then compare 
that to the fragmentation of the first fit solution. 

\section{Version Control Log}
\begin{tabular}{l l l}\textbf{Detail} & \textbf{Author} & \textbf{Description}\\\hline
\href{https://github.com/crosssh/CS444/commit/77ef8915c0518852fc5179cfdcd9d13c765f43b0}{77ef891} & Shawn Cross & adding concurrency 3 files.\\\hline
\href{https://github.com/crosssh/CS444/commit/493293c4ed6f02fa0ae80b534699703d705665b9}{493293c} & Shawn Cross & adding last files for concurrency 3\\\hline
\href{https://github.com/crosssh/CS444/commit/a36cf1d987dc262bd9055c419b370a4d2b430c38}{a36cf1d} & Shawn Cross & adding the title page to the tex document\\\hline
\href{https://github.com/crosssh/CS444/commit/a488836077a629fcc0e9ef43ac11cec51df94038}{a488836} & Shawn Cross & fixed small problem\\\hline
\href{https://github.com/crosssh/CS444/commit/99fd75bdfb7c68d9498469a85f246ba4db9aa885}{99fd75b} & Shawn Cross & Adding assignment 4 forgot to commit the previous changes.\\\hline\end{tabular}

\section{Work Log}
Tuesday November 28th 10 - 11:30am\\ 
Began the assignment modified slob.c with a best fit solution,
but couldn't make the kernel run it.\\\\
Wednesday November 29th 1:30pm - 3:30pm\\
Went to office hours for questions, got kernel to load slob code.\\\\
Thursday November 30th 8:30am - 11:30am\\
implemented best fit in the kernel and wrote the system calls.
Began the test program to check memory use.\\\\
Friday December 1st 9:00am - 4:00pm\\
Finished testing program and wrote tex document.
\section{Project Questions}

\subsection{Main point of the assignment}
The main point of this assignment is to teach us about how memory is managed 
in the linux kernel as well as teach us that some times a solution that sounds
the best is not always the right solution.

\subsection{How did we approach the problem}
We knew we needed to edit the current slob file to use best fit instead of first 
fit. So we decide to try and figure out how to do this first. We came up with a 
simple solution. Set our page struct equal to the orignal one and then check if 
the next one is smaller if it is set our equal to it. Then check if the current 
spot is equal to out chunk and if it is put it in that spot if not then try the 
next spot. Continue doing this until we find a spot that is either the same or 
the next page is NULL. In this case you would put it in the next best spot 
or create a new page. Then for the  system call we just find the number of 
pages being used and then multiple it by the size of a page. For the free 
space we loop thorough all the pages finding all the free units and then add them 
together. We then us these to find the percentage of fragmentation.

\subsection{Ensure correctness and details of how to preform}
\section{TA Evaluation}
Steps for applying patch to clean linux-yocto-3.19 tree.\\\\
\begin{enumerate}
\item{step 1: }
   clone new repo from github using the command\\\\
   git clone git://git.yoctoproject.org/linux-yocto-3.19
\item{step 2: }
   switch to the 3.19.2 by using the command\\\\
   git checkout -b linux-yocto-3.19.2
\item{step 3: }
   copy the file /scratch/fall2017/files/config-3.19.2-yocto-qemu into a file 
   named .config into the source root of the repo that you just cloned.
\item{step 4: }
   Now source the environment configuration. If you are using bash you will 
   need to source /scratch/fall2017/files/environment-setup-i586-poky-linux 
   and if you are using tcsh you will need  to source 
   /scratch/fall2017/files/environment-setup-i586-poky-linux.csh
\item{step 5: }
   You need to apply the patch we provided to the repo by using the command\\\\
   git apply [path to patch file] where path to patch file is the path you need 
   to the patch file that we have provided. We have provided two patches 
   firstfit.patch will be for the original slob that comes with the fresh kernel 
   and then adds the system call function for testing. The bestfit.patch 
   is our best fit solution along with the system call stuff as well. You can 
   apply either wbut you will have to go back and clone a new repo before testing 
   the other. 
\item{step 6: } 
   Yow will need to now run the command make menuconfig. Once this is done a 
   menu will open and you will need to scroll down and change from using slub 
   to using slob. This is in general setup towards the botton in the option 
   choose slab allocator.
\item{step 7: }
   You now need to make the kernel by running the command make -j4 
\item{step 8: }
   Once the kernel is created you will need to run the vm using the command\\\\
   qemu-system-i386 -gdb tcp::5550 -S -nographic -kernel [path to bzImage] -drive file=[path to core image],if=virtio -enable-kvm -usb -localtime --no-reboot --append "root=/dev/vda rw console=ttyS0 debug"\\\\
   Where [path to core image] is the file found in\\\\ 
   /scratch/fall2017/files/core-image-lsb-sdk-qemux86.ext4\\\\
   and the [path to bzImage] for me would be the bzImage found in\\\\
   linux-yocto-3.19/arch/x86/boot/bzImage\\\\
   and the ???? are the port number that you want to connect gdb to remotely.\\\\
   After this command is run the terminal will halt.
\item{step 9: }
   you will now need to run \$GDB in a new terminal window. You may need soucre 
   the enviroment again as you did in step 4 previously if you did start a new 
   termial session and the \$GDB command does not work. Once gdb is running you 
   will use the command\\\\
   target remote :????\\\\
   where the ???? are the same as the port that you had specified in step 7.\\\\
   then type continue and the first terminal should start.
\item{step 10:}
   You need to then scp the the page\_test.c and the mt19937.h files that we provided 
   over to your running vm 

\item{step 11:}
   You will then need to comiple this program by using the command: gcc page\_test.c\\
   then you can run it by running: ./a.out\\\\
   this will output a bunch of precentages. These precentages are the amount of free 
   free space avalible. The program mallocs and free's a bunch of random size arrays.
   You will notice that as the malloc's and free's happen you will see that we the 
   precentages are increasing and decreasing. This goes along with the idea that as 
   we are using more space the amout of free space will decreas and as we free the 
   amount of space will increase. There are some random flucuation we believe this is 
   because as pages are added or removed the amount of free space would increase 
   quite a bit from this.

\item{step 12:}
   Now you will need to repeat theses steps to use the other patch that we have 
   provided. You can either revert the repo back to its original state and then 
   apply the other patch or you can download a fresh repo. Once you have does this 
   you can start from either step on if you want a fresh repo or step two if you 
   reverted the repo.

\item{Best fit Vs First fit: }
   One thing we did notice is that with the first fit solution the percentage of 
   free space is much less than what we got for when we applied our best fit 
   soultion. we are not exactly sure why this is but we did notice that our 
   solution is flucuating the same way that the first fit is. We believe that 
   this shows that our solution is in fact working and that the space is being 
   allocated the way that it is supposed to be.
   
\end{enumerate}
\subsection{What did we learn}
During this project we learned that while some solution seem to be the better 
idea it doesn't mean that they are the better way to do it. while first fit 
does seem to use a little more space and leaves more small holes in each page 
it also doesn't take as long to do. Our version of best fit causes adding and 
freeing space to take a lot more time to a point where the outcome is not a 
benifit. 

\end{document}
