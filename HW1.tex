\documentclass[letterpaper,10pt,titlepage]{article}

\usepackage{graphicx}                                        
\usepackage{amssymb}                                         
\usepackage{amsmath}                                         
\usepackage{amsthm}                                          

\usepackage{alltt}                                           
\usepackage{float}
\usepackage{color}
\usepackage{url}

\usepackage{balance}
%\usepackage[TABBOTCAP, tight]{subfigure}
\usepackage{enumitem}
%\usepackage{pstricks, pst-node}

\usepackage{geometry}
\geometry{textheight=8.5in, textwidth=6in}

\newcommand{\cred}[1]{{\color{red}#1}}
\newcommand{\cblue}[1]{{\color{blue}#1}}

\usepackage{hyperref}
\usepackage{geometry}

\def\name{Ryan Crane and Shawn Cross}

%pull in the necessary preamble matter for pygments output
%\input{pygments.tex}

%% The following metadata will show up in the PDF properties
\hypersetup{
  colorlinks = true,
  urlcolor = black,
  pdfauthor = {\name},
  pdfkeywords = {cs444 ``operating systems'' files filesystem I/O},
  pdftitle = {CS 444 Project 1: UNIX File I/O},
  pdfsubject = {CS 444 Project 1},
  pdfpagemode = UseNone
}

\begin{document}

\section{Log of Commands}
Fist we cloned the linux-yocto-3.19 git hub repo into our group 50 directory. 
We then sourced the directory using\\source ../../files/environment-setup-
i586-poky-linux\\then we moved the config file into the yocto directoy 
using the command\\cp ../../../files/config-3.19.2-standard .config\\ and 
then ran the command\\make menuconfig\\ to change the local version name.
after we had that setup we then tested that the kernel worked by coping the
bzImage-qemux86.bin file and the core-image-lsb-sdk-qemux86.ext4 file into
our yocto directory. we then ran the command\\qemu-system-i386 -gdb tcp::5550
-S -nographic -kernel bzImage-qemux86.bin -drive 
file=core-image-lsb-sdk-qemux86.ext4,if=virtio -enable-kvm -net none -usb 
-localtime --no-reboot --append "root=/dev/vda rw console=ttyS0 debug"\\ to 
start the vm in a halted state. Next we used gdb to connect to the vm remotly 
by using the commands\\\$GDB\\target remote os2:5550\\Finally we built our 
kernel by running\\make -j4 all\\
and then test that it worked by running the vm again and using the new bzImage 
qemu-system-i386 -gdb tcp::???? -S -nographic -kernel  arch/x86/boot/bzImage 
-drive file=core-image-lsb-sdk-qemux86.ext4,if=virtio -enable-kvm -net none 
-usb -localtime --no-reboot --append "root=/dev/vda rw console=ttyS0 debug"\\
and then re-ran gdb to connect remotly.



\section{Concurrency Solution}
We create ten threads, five producer threads, and five comsumer threads. The
producer thread will call the function producer\_start that will run forever.
In the function we get a random value and sleep for that amount of time. We then
get two random values one of which we use to determine the amount of time that
the consumer will sleep for and the other is the value that the consumer will
print out after the consumption time. We then use a semaphore to keep track 
of the space availible in the buffer. If the space becomes full this will 
block the producer until another space becomes avalible. Next we use a mutex
to lock the other treads out while we put the item into the buffer. Finally we
will increment the index of the buffer, unlock the buffer, and increment an 
items semaphore that will keep track of wether or not there are any items are 
in the buffer and if none are it will block. When we create the consumer 
threads we call the consumer\_start. This function will first decrement the 
items semaphore, lock the mutex, get the item out of the buffer, decrement the 
index of the buffer, unlock the mutex and increment the space semaphore, and 
finally sleep for the consumption time and prints the value and frees the 
memory. We also have an infinite loop at the end of the main function so the 
program will not end.

\section{List of Commands}
\begin{enumerate}
   \item[-nographic] disables the graphical output and redirects serial I/Os the console
   \item[-kernel] uses a 'bzImage' as the kernel image.
   \item[-drive] uses a 'file' as a drive image.
   \item[-enable] enable KVM full virtualization support.
   \item[-net none] use it alone to have zero network devices.
   \item[-usb] enable the USB driver.
   \item[-localtime] 
   \item[--no-reboot] exit instead of rebooting.
   \item[-append] cmdline use 'cmdline' as kernel command line
\end{enumerate}

\section{Point of the assignment}
To understand how multiple threads working at the same time need to use
synchronization when they are sharing data the can be manipulated by each of
the threads indivually.

\section{how was the problem approached}
We first read how to solve the problem in the Little Book Of Semaphores. we 
then looked up the semaphores syntax in c. We also looked pthread syntax. We 
first got the random numbers to generate using one of the options avalible. 
We then created the struct and the pthreads. We got the producer function 
working and then we got the consumer function working.

\section{How did we test the solution}
We added in print statements and minipulated the number of producers and 
consumeres to make sure that the semaphores were properly blocking. We 
made an abundent amount of producers and a small amount of consumers. Initially 
the program had a lot of statements being printed by all of the produce threads 
and eventually the number of print statements decreased and got to a point where 
they only printed after a consumer message printed proving that the producer 
was being blocked until a consumer removed something from the buffer. To test 
that the semaphore was blocking when there was nothing in the buffer we 
preformed the same test except we had an abundant amount of consumers and a 
small amount of producers. 

\section{What did we learn}
That the semaphores are very useful tools for helping with synchronization 
across multiple threads. Type declarations can be very important. We 
accidentally had the get random number fuction returning the wrong type and 
it caused our whole program to not run the way we expected. 

\section{Version control log}
\begin{center}
   \begin{tabular}{ |c|c|c|c|}
      \hline
      commit 1&commit 2&commit 3&commit 4\\ 
      \hline
      Author: Shawn Cross &Author: Ryan Crane &Author: Shawn Cross &Author: Shawn Cross \\
      \hline
      Date:   2017-10-08&Date:   2017-10-08&Date:   2017-10-08&Date:   2017-10-08\\
      \hline
      initial commit&concurrency&updating the tex file And adding the pdf&changes to the tex file.\\
      \hline
   \end{tabular}
\end{center}
%input the pygmentized output of mt19937ar.c, using a (hopefully) unique name
%this file only exists at compile time. Feel free to change that.
%\input{__mt19937ar.c.tex}
\end{document}
