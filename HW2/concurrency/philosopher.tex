\documentclass[letterpaper,10pt,titlepage]{article}

\usepackage{graphicx}                                        
\usepackage{amssymb}                                         
\usepackage{amsmath}                                         
\usepackage{amsthm}                                          

\usepackage{alltt}                                           
\usepackage{float}
\usepackage{color}
\usepackage{url}

\usepackage{balance}
%\usepackage[TABBOTCAP, tight]{subfigure}
\usepackage{enumitem}
%\usepackage{pstricks, pst-node}

\usepackage{geometry}
\geometry{textheight=8.5in, textwidth=6in}

\newcommand{\cred}[1]{{\color{red}#1}}
\newcommand{\cblue}[1]{{\color{blue}#1}}

\usepackage{hyperref}
\usepackage{geometry}

\def\name{Ryan Crane and Shawn Cross}

%pull in the necessary preamble matter for pygments output
%\input{pygments.tex}

%% The following metadata will show up in the PDF properties
\hypersetup{
  colorlinks = true,
  urlcolor = black,
  pdfauthor = {\name},
  pdfkeywords = {cs444 ``operating systems'' files filesystem I/O},
  pdftitle = {CS 444 Project 1: UNIX File I/O},
  pdfsubject = {CS 444 Project 1},
  pdfpagemode = UseNone
}

\begin{document}

\title{Philosopher Problem}
\date{October 26, 2017}
\author{Shawn Cross, Ryan Crane}
\maketitle

To compile the program\\type: make noodles\\\\To run the program\\type: noodles
\\\\To end the program press: Crtl-c\\\\

To figure out a solution to this problem we used the little book of semaphores. 
specifically we used the footman solution. First we created the functions for 
thinking and eating and used the Mersenne Twister for getting the random times. 
Once we had the functions created we made the threads for the philosophers and 
the fork and footman semaphores. We first just created one start function for 
all of the theads but then realized that we would need to create one for each 
since they all needed to be numbered and nameded. These funcion consist of a 
while loop that will run forever and reapatedly call the think, pick up fork, 
eat, and put down fork functions. once we had these funcitons we created the 
functions for picking up and putting down the forks which is the same a the 
solution in the litlle book of semaphores and is to wait on the footman and 
then to wait on the right fork and then the left. The put down function is 
the exact opposite uses post on the left and right fork and then post on the 
footman. The forks are an array of semaphores that are all initially set to 
one so when a philosopher use wait on the fork it will block until they are 
done eating and then put it back down. The footman is initialiezd to four and 
will block before all five people could pick up a fork keeping there from being 
a deadlock.\\ 

We can see that given a time frame between 30 and 60 seconds that all of the 
philosophers are eatting at least one time and no fork is ever being picked 
up by two philosophers at the same time either. 

\end{document}
