\documentclass[letterpaper,10pt,titlepage]{article}

\usepackage{graphicx}                                        
\usepackage{amssymb}                                         
\usepackage{amsmath}                                         
\usepackage{amsthm}                                          

\usepackage{alltt}                                           
\usepackage{float}
\usepackage{color}
\usepackage{url}

\usepackage{balance}
%\usepackage[TABBOTCAP, tight]{subfigure}
\usepackage{enumitem}
%\usepackage{pstricks, pst-node}

\usepackage{geometry}
\geometry{textheight=8.5in, textwidth=6in}

\newcommand{\cred}[1]{{\color{red}#1}}
\newcommand{\cblue}[1]{{\color{blue}#1}}

\usepackage{hyperref}
\usepackage{geometry}

\def\name{Ryan Crane and Shawn Cross}

%pull in the necessary preamble matter for pygments output
%\input{pygments.tex}

%% The following metadata will show up in the PDF properties
\hypersetup{
  colorlinks = true,
  urlcolor = black,
  pdfauthor = {\name},
  pdfkeywords = {cs444 ``operating systems'' files filesystem I/O},
  pdftitle = {CS 444 Project 1: UNIX File I/O},
  pdfsubject = {CS 444 Project 1},
  pdfpagemode = UseNone
}

\begin{document}

\title{Assignment 3}
\date{October 13, 2017}
\author{Shawn Cross, Ryan Crane}
\maketitle

\section{Plan for Implementation}

For this project we plan on first finding a simple block device similar to the 
one in the Linux Device Drivers book that will work with our kernel. Once we 
have found one that works with our kernel we will then add the cryptography to 
it. 

\section{Version Control Log}

\section{Work Log}

\section{Project Questions}

\subsection{Main point of the assignment}
The main point of this assignment was to learn how linux modules and device drivers work, 
as well as gaining some experience with linux's crypto api. 
\subsection{How did we approach the problem}
We began by finding a working block device driver to use as a base. Then we looked for examples 
where linux's cryptographic tools were used to determine how to add encryption to the device. We chose
 to use AES as our block cipher with a default key of all zeros when the module parameter is blank.
\subsection{Ensure correctness and details of how to preform}
\section{TA Evaluation}
Steps for applying patch to clean linux-yocto-3.19 tree.\\\\
\begin{enumerate}
\item{step 1: }
   clone new repo from github using the command\\\\
   git clone git://git.yoctoproject.org/linux-yocto-3.19
\item{step 2: }
   switch to the 3.19.2 by using the command\\\\
   git checkout -b linux-yocto-3.19.2
\item{step 3: }
   copy the file /scratch/fall2017/files/config-3.19.2-yocto-qemu into a file 
   named .config into the source root of the repo that you just cloned.
\item{step 4: }
   Now source the environment configuration. If you are using bash you will 
   need to source /scratch/fall2017/files/environment-setup-i586-poky-linux 
   and if you are using tcsh you will need  to source 
   /scratch/fall2017/files/environment-setup-i586-poky-linux.csh
\item{step 5: }
   You need to apply the patch we provided to the repo by using the command\\\\
   git apply [path to patch file] where path to patch file is the path you need 
   to the patch file that we have provided.
\item{step 6: }
   You now need to make the kernel by running the command make -j4 
\item{step 7: }
   Once the kernel is created you will need to run the vm using the command\\\\
   qemu-system-i386 -gdb tcp::5550 -S -nographic -kernel [path to bzImage] -drive file=[path to core image],if=virtio -enable-kvm -usb -localtime --no-reboot --append "root=/dev/vda rw console=ttyS0 debug"\\\\
   Where [path to core image] is the file found in\\\\ 
   /scratch/fall2017/files/core-image-lsb-sdk-qemux86.ext4\\\\
   and the [path to bzImage] for me would be the bzImage found in\\\\
   linux-yocto-3.19/arch/x86/boot/bzImage\\\\
   and the ???? are the port number that you want to connect gdb to remotely.\\\\
   After this command is run the terminal will halt.
\item{step 8: }
   you will now need to run \$GDB in a new terminal window. You may need soucre 
   the enviroment again as you did in step 4 previously if you did start a new 
   termial session and the \$GDB command does not work. Once gdb is running you 
   will use the command\\\\
   target remote :????\\\\
   where the ???? are the same as the port that you had specified in step 7.\\\\
   then type continue and the first terminal should start.
\item{step 9:}
   You need to then scp the sbull.ko file in and then scp the provided script test.sh.

   once you have both in the vm run ./test.sh this will set up and mount the module to
   a directory named tempmount. The script can take a little bit of time.

   if the test script doesn't work here are the stpes to follow.\\

   first: insmod sbull.ko\\
   second: fdisk /dev/sbulla\\
   third: when promted in fdisk press g then enter then n then enter until it prompts 
   for another letter and then press w and then enter. This will create a partition.\\
   forth: mkfs.ext2 /dev/sbulla\\
   fifth: mkdir tempmount\\
   sixth: mount /dev/sbulla1 /tempmount\\
\item{step 10:}
   Type echo 'hello world' > tempmount/hello
   
   When you do this it will show two longer number one after the other. This can also 
   happen mutiple times depending on the size of what is being passed to the module.
   The fisrt number is the data before it is encrypted and the second is the encryped 
   data. They are different which shows that the encryption worked. 
   
   Then type cat tempmount/hello

   This should show the what was originally passed in in it's unecrypted form 
   showing that the decryption has worked. 
   
   
\end{enumerate}
\subsection{What did we learn}
We learned that all linux documentation is terrible. We also learned about manipulating 
block ciphers in linux and setting module parameters, and loading them into linux. we also had
to mount a block device and create a filesystem for it.

\end{document}
