\documentclass[letterpaper,10pt,titlepage]{article}

\usepackage{graphicx}                                        
\usepackage{amssymb}                                         
\usepackage{amsmath}                                         
\usepackage{amsthm}                                          

\usepackage{alltt}                                           
\usepackage{float}
\usepackage{color}
\usepackage{url}

\usepackage{balance}
%\usepackage[TABBOTCAP, tight]{subfigure}
\usepackage{enumitem}
%\usepackage{pstricks, pst-node}

\usepackage{geometry}
\geometry{textheight=8.5in, textwidth=6in}

\newcommand{\cred}[1]{{\color{red}#1}}
\newcommand{\cblue}[1]{{\color{blue}#1}}

\usepackage{hyperref}
\usepackage{geometry}

\def\name{Ryan Crane and Shawn Cross}

%pull in the necessary preamble matter for pygments output
%\input{pygments.tex}

%% The following metadata will show up in the PDF properties
\hypersetup{
  colorlinks = true,
  urlcolor = black,
  pdfauthor = {\name},
  pdfkeywords = {cs444 ``operating systems'' files filesystem I/O},
  pdftitle = {CS 444 Project 1: UNIX File I/O},
  pdfsubject = {CS 444 Project 1},
  pdfpagemode = UseNone
}

\begin{document}

\title{Assignment 3}
\date{October 13, 2017}
\author{Shawn Cross, Ryan Crane}
\maketitle

\section{Plan for Implementation}

For this project we plan on first finding a simple block device similar to the 
one in the Linux Device Drivers book that will work with our kernel. Once we 
have found one that works with our kernel we will then add the cryptography to 
it. 

\section{Version Control Log}

\section{Work Log}

\section{Project Questions}

\subsection{Main point of the assignment}

\subsection{How did we approach the problem}

\subsection{What did we learn}

\end{document}
