\documentclass[letterpaper,10pt,titlepage]{article}

\usepackage{graphicx}                                        
\usepackage{amssymb}                                         
\usepackage{amsmath}                                         
\usepackage{amsthm}                                          

\usepackage{alltt}                                           
\usepackage{float}
\usepackage{color}
\usepackage{url}

\usepackage{balance}
%\usepackage[TABBOTCAP, tight]{subfigure}
\usepackage{enumitem}
%\usepackage{pstricks, pst-node}

\usepackage{geometry}
\geometry{textheight=8.5in, textwidth=6in}

\newcommand{\cred}[1]{{\color{red}#1}}
\newcommand{\cblue}[1]{{\color{blue}#1}}

\usepackage{hyperref}
\usepackage{geometry}

\def\name{Ryan Crane and Shawn Cross}

%pull in the necessary preamble matter for pygments output
%\input{pygments.tex}

%% The following metadata will show up in the PDF properties
\hypersetup{
  colorlinks = true,
  urlcolor = black,
  pdfauthor = {\name},
  pdfkeywords = {cs444 ``operating systems'' files filesystem I/O},
  pdftitle = {CS 444 Project 1: UNIX File I/O},
  pdfsubject = {CS 444 Project 1},
  pdfpagemode = UseNone
}

\begin{document}

\title{Assignment 3}
\date{October 13, 2017}
\author{Shawn Cross, Ryan Crane}
\maketitle

\section{Plan for Implementation}

For this project we plan on first finding a simple block device similar to the 
one in the Linux Device Drivers book that will work with our kernel. Once we 
have found one that works with our kernel we will then add the cryptography to 
it. 

\section{Version Control Log}

\section{Work Log}

\section{Project Questions}

\subsection{Main point of the assignment}
The main point of this assignment was to learn how linux modules and device drivers work, 
as well as gaining some experience with linux's crypto api. 
\subsection{How did we approach the problem}
We began by finding a working block device driver to use as a base. Then we looked for examples 
where linux's cryptographic tools were used to determine how to add encryption to the device. We chose
 to use AES as our block cipher with a default key of all zeros when the module parameter is blank.

\subsection{What did we learn}
We learned that all linux documentation is terrible. We also learned about manipulating 
block ciphers in linux and setting module parameters, and loading them into linux. we also had
to mount a block device and create a filesystem for it.

\end{document}
