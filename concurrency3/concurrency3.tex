\documentclass[letterpaper,10pt,titlepage]{article}

\usepackage{graphicx}                                        
\usepackage{amssymb}                                         
\usepackage{amsmath}                                         
\usepackage{amsthm}                                          

\usepackage{alltt}                                           
\usepackage{float}
\usepackage{color}
\usepackage{url}

\usepackage{balance}
%\usepackage[TABBOTCAP, tight]{subfigure}
\usepackage{enumitem}
%\usepackage{pstricks, pst-node}

\usepackage{geometry}
\geometry{textheight=8.5in, textwidth=6in}

\newcommand{\cred}[1]{{\color{red}#1}}
\newcommand{\cblue}[1]{{\color{blue}#1}}

\usepackage{hyperref}
\usepackage{geometry}

\def\name{Ryan Crane and Shawn Cross}

%pull in the necessary preamble matter for pygments output
%\input{pygments.tex}

%% The following metadata will show up in the PDF properties
\hypersetup{
  colorlinks = true,
  urlcolor = black,
  pdfauthor = {\name},
  pdfkeywords = {cs444 ``operating systems'' files filesystem I/O},
  pdftitle = {CS 444 Project 1: UNIX File I/O},
  pdfsubject = {CS 444 Project 1},
  pdfpagemode = UseNone
}

\begin{document}

\title{CS444 Concurrency 3}
\date{November 26, 2017}
\author{Shawn Cross and Ryan Crane}

\section{How To Run} To compile the program simply 

run: make all

Then to run the first problem 

run: problem1

Finally to run the second problem 

run: list

\section{Problem 1.} For the first problem we found that this was very similar 
to the sushi bar problem in the little book of semaphores. We followed the 
psudo code that they gave as a solution. We create seven threads that will 
run forever. In the function we have a must\_wait variable that is false if 
less than three processes are using a resource and ture if three are. If this is set to 
true then then processes will be blocked in the must wait function until all 3 processes 
are finished. We have an if statement that checks if you are the last person 
eating and if the must\_wait flag is set, if both are true then we unlock the 
mutex that is blocking the rest of the processes and allow them to go. This 
program will run forever but looking at the outputs will show that the 
program is working correctly. You will see that when there are less than 
three eating processes are coming and going freely. However, when there 
are three processes using the resources then everyone else will be waiting 
until the three processes have finished and then others will start to eat again.

To end this program use ctrl-c. It also may take awhile for the program to 
output processes coming and going freely but it shouldn't take more than 
two minutes.

\section{Problem 2.} For this problem we first created the linked list functions 
that we were going to need, Insert, pop, and delete. Next we created the 
ligthswitch functions that the book suggested we use. These check if you are 
the first searcher and if you are then lock the semaphore and keep track of 
the number of searchers. If you are the last searcher to leave then unlock 
the semaphore. This is similar for the inserter as well. For our searchers 
function we first lock the lightswitch and then get a random value to enter. 
next we search through the list and if the value is found we say we found it, 
and if not we say we didn't find it, then we unlock the lightswitch. For 
inserters we first get a random number to insert and then lock our lightswitch 
we also block any other inserters from getting in as well. The value is then 
inseted into the list the size of the list is increased and we print what was 
inserted. The mutex blocking other inserters is unlocked and the lightswitch is 
is also unlocked. Finally the deleter function checks the size of the list and 
then from that picks a random index between 0 and size to delete. It then trys 
to wait on the insert and search lightswitch semaphores and can only delete 
if neither of them are blocking. If neither are then it waits on both causing 
them to block everyone else out, deletes the index, prints what was deleted, 
and then finally unblocks everyone. At the start of the program we create an 
initial list to use. we then create 10 of each type of thread and then they 
run. You can see for the outputs of this program that while searching is 
happening an insert is still able to happen but no deletes. and you will 
also notice that only the delete is allowed to happen and searchers and 
inserters aren't allowed to do anything.

\end{document}
