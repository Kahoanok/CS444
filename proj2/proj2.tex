\documentclass[letterpaper,10pt,titlepage]{article}

\usepackage{graphicx}                                        
\usepackage{amssymb}                                         
\usepackage{amsmath}                                         
\usepackage{amsthm}                                          

\usepackage{alltt}                                           
\usepackage{float}
\usepackage{color}
\usepackage{url}

\usepackage{balance}
\usepackage[TABBOTCAP, tight]{subfigure}
\usepackage{enumitem}
\usepackage{pstricks, pst-node}

\usepackage{geometry}
\geometry{textheight=8.5in, textwidth=6in}

\newcommand{\cred}[1]{{\color{red}#1}}
\newcommand{\cblue}[1]{{\color{blue}#1}}

\usepackage{hyperref}
\usepackage{geometry}

\def\name{D. Kevin McGrath}

%pull in the necessary preamble matter for pygments output
%\input{pygments.tex}

%% The following metadata will show up in the PDF properties
\hypersetup{
  colorlinks = true,
  urlcolor = black,
  pdfauthor = {\name},
  pdfkeywords = {cs444 ``operating systems'' files filesystem I/O},
  pdftitle = {CS 444 Project 1: UNIX File I/O},
  pdfsubject = {CS 444 Project 1},
  pdfpagemode = UseNone
}

\begin{document}

\title{CS4444 Project 2}
\date{October 30, 2017}
\author{Shawn Cross, Ryan Crane}
\maketitle

We plan to modified the the dispatch and add functions from the original noop 
scheduler to implement the LOOK algorithm for the scheduler. We made it so that 
you can go both forward and backward in the dispatch function when finding the 
next request to dispatch. If the distance to a request in the opposite 
direction is closer than a request in the current direction the direction will 
change. The queue sorted after a request has been dispatched. If the queue is 
empty then we add the new requests directly to the queue. If there is already 
requests in the queue then when adding then we sort during the insertion. the 
requests are inserted into the queue based on position. 

\section{Version Control log}
\begin{tabular}{l l l}\textbf{Detail} & \textbf{Author} & \textbf{Description}\\\hline
\href{https://github.com/crosssh/CS444/commit/88e93dc7f5e10022f61dcc02707ab977045d02f0}{88e93dc} & Ryan Crane & dining philosophers problem, looks correct\\\hline
\href{https://github.com/crosssh/CS444/commit/35c460c5834e37d3b7763cb8cf924fd1c60605aa}{35c460c} & Shawn Cross & adding the write up for the concurrency problem.\\\hline
\href{https://github.com/crosssh/CS444/commit/3f0111c4029242fbff1abdbca709846e312de5e3}{3f0111c} & Shawn Cross & More done on the write up and made it so you can compile for the makefile.\\\hline
\href{https://github.com/crosssh/CS444/commit/2d479602bacb4413c4926dc4d291da1999f4543c}{2d47960} & Shawn Cross & slight change to tex file.\\\hline
\href{https://github.com/crosssh/CS444/commit/9aa562c78ee7ab1d2c149a0a110375449b559d34}{9aa562c} & narxy & Update philosopher.tex\\\hline
\href{https://github.com/crosssh/CS444/commit/f8b1617c75f383036ef25a3ac7a69fd1960cee15}{f8b1617} & narxy & Update philosopher.tex\\\hline
\href{https://github.com/crosssh/CS444/commit/651df07401865b4a7a8b049a5bb170d350407c4d}{651df07} & Shawn Cross & fixed spelling errors.\\\hline
\href{https://github.com/crosssh/CS444/commit/75556889aec755f28b25f1c81b86fb31c7ff825c}{7555688} & Shawn Cross & uploading tarball and created the sstf-iosched.c file\\\hline
\href{https://github.com/crosssh/CS444/commit/4bc64ff20f38125e7a8d84f16e1780b235ff3519}{4bc64ff} & Shawn Cross & Uptdate to the Kconfig file, update to the Makefile, update to menuconfig, and sstf-iosched.c added\\\hline\end{tabular}


\section{Work Log}
Thursday 10/26 6:00 pm to 8:00 pm\\\\
started assignment mostly reading up on Kernel.\\\\ 
Friday 10/27 12:00 pm to 4:00 pm \\\\
modified config and makefiles to adopt our new scheduler.\\\\
Sunday 10/29 12:00 pm to 5:00 pm \\\\
Implemented Look algorithm and started the \LaTeX document.\\\\
Monday 10/9 12:00 pm to 2:00 pm \\\\
Finished the \LaTeX document.\\\\

\section{Main point of the assignment}
We believe that the main point of the assignment was to understand how requests 
are dealt with in an operating system. It was also begin to understand how to 
interact with the kernel. 

\section{How we approached the problem}
First we started by figuring out how to get our scheduler working with the kernel 
instead of the noop scheduler. Once we were able to do that we began to figure 
out how we were going to implement the look schedluer. We knew we needed to be 
able move forward and backward and that there needed to be two sorts. First we 
firgured out how to get the forward and backward working by checking if the the 
next request in the direction that we were currently going is closer than the 
next request in the oppsite direction. If it was closer than we continue in that 
direction and if it was futher away then we change direction and go to the closer 
request. After a request is dispatched then we will re-sort the queue. We also 
needed to modify the add function so that we would sort the requests as they were 
inserted into the queue. We did this by using an insertion sort as requests were 
added into the queue.

\section{Correctness and Testing}
We know that our secheduler was being used because we used the command

cat /sys/block/hda/queue/scheduler

and got an output

noop [sstf] deadline cfq

this shows that our sstf scheduler is being used as we wanted. We then 
added print statements to our scheduler code so that we could determine 
when requests were being added and dispatched from the queue and by doing 
this we were able to show that for every requeust that was being added to 
the queue it was also being dispatched as well which means that all requests 
were being dealt with. We also added print statements to declare when the direction flag had its value changed to show that we were in fact changing direction. With all of these things 
working as we expected them to we were able to prove that our look 
scheduler was properly working the way we wanted it to. 

\section{What we learned}
We learned that you have to make modification to multiple files when adding 
new features to the operating system. We also learned that writing code for 
the kernel can be hard and confusing because we could not find a lot of 
information on some of the built-in function being used so it took us some 
time trying to figure what they did and how they worked. We also learned that 
can be a ton of I/O requests and they can all happen very quickly. We also 
learned that there are many different ways to do the scheduling. 

\section{TA Evaluation}
Steps for applying patch to clean linux-yocto-3.19 tree.\\\\
\begin{enumerate}
\item{step 1: }
   clone new repo from github using the command\\\\
   git clone git://git.yoctoproject.org/linux-yocto-3.19
\item{step 2: }
   switch to the 3.19.2 by using the command\\\\
   git checkout -b linux-yocto-3.19.2
\item{step 3: }
   copy the file /scratch/fall2017/files/config-3.19.2-yocto-qemu into a file 
   named .config into the source root of the repo that you just cloned.
\item{step 4: }
   Now source the environment configuration. If you are using bash you will 
   need to source /scratch/fall2017/files/environment-setup-i586-poky-linux 
   and if you are using tcsh you will need  to source 
   /scratch/fall2017/files/environment-setup-i586-poky-linux.csh
\item{step 5: }
   You need to apply the patch we provided to the repo by using the command\\\\
   git apply <path to patch file>
\item{step 6: }
   You now need to make the kernel by running the command make -j4 
\item{step 7: }
   Once the kernel is created you will need to run the vm using the command\\\\
   qemu-system-i386 -gdb tcp::???? -S -nographic -kernel <path to bzImage> -drive file=<path to core image> -enable-kvm -net none -usb -localtime --no-reboot --append "root=/dev/hda rw console=ttyS0 debug\\\\
   Where <path to core image> is the file found in\\\\ 
   /scratch/fall2017/files/core-image-lsb-sdk-qemux86.ext4\\\\
   and the path to bzImage for me would be the bzImage found in\\\\
   linux-yocto-3.19/arch/x86/boot/bzImage\\\\
   and the ???? are the port number that you want to connect gdb to remotely.\\\\
   After this command is run the terminal will halt.
\item{step 8: }
   you will now need to run \$GDB in a new terminal window. You may need soucre 
   the enviroment again as you did in step 4 previously if you did start a new 
   termial session and the \$GDB command does not work. Once gdb is running you 
   will use the command\\\\
   target remote :????\\\\
   where the ???? are the same as the port that you had specified in step 7.
\item{step 9: }
   Once you have gdb is remotely connected you can type continue and the terminal 
   with the vm halted will begin to boot. You will be asked if you want to use the 
   sstf scheduler. You should be able to press y and then enter and and you will 
   be promted about which scheduler to use again. I beleive our is option 4 but 
   pick the number that is next to the sstf option. The booting will continue 
   after this and eventually you will see our print statements about requests being 
   added and dispatched. You should evtually see a login promt but you may also miss 
   it from all of the print statements. Wait a few seconds and then eventually the 
   print statements will stop. type root to login and you will begin to see more 
   print statement. You will notice that some will have first request added and 
   then immidiatley see a statemet about the request being dispatched. You will 
   then see points at which multiple requests are added and then see that our 
   scheduler is in fact moving forward and backward. and eventually the statements 
   will stop coming and you can then look and see that our schedular is working as 
   we had wanted it to by examining the previous statements that were outputted to 
   the terminal.
\item{step 10: }
   To shutdown the vm you simply need to type reboot and press enter. Once the 
   vm has shut down you can then exit gdb by typing quit into the terminal that 
   you have gdb open in.
   
\end{enumerate}

\end{document}
